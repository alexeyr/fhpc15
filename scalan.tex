\documentclass[preprint]{sigplanconf}

% The following \documentclass options may be useful:

% preprint      Remove this option only once the paper is in final form.
% 10pt          To set in 10-point type instead of 9-point.
% 11pt          To set in 11-point type instead of 9-point.
% authoryear    To obtain author/year citation style instead of numeric.

\usepackage{amsmath}

\begin{document}

\special{papersize=8.5in,11in}
\setlength{\pdfpageheight}{\paperheight}
\setlength{\pdfpagewidth}{\paperwidth}

\conferenceinfo{FHPC '15}{September 3, 2015, Vancouver, Canada} 
\copyrightyear{2015} 
\copyrightdata{978-1-nnnn-nnnn-n/yy/mm} % TODO
\doi{nnnnnnn.nnnnnnn} % TODO

% Uncomment one of the following two, if you are not going for the 
% traditional copyright transfer agreement.

%\exclusivelicense                % ACM gets exclusive license to publish, 
                                  % you retain copyright

%\permissiontopublish             % ACM gets nonexclusive license to publish
                                  % (paid open-access papers, 
                                  % short abstracts)

\titlebanner{DRAFT}        % These are ignored unless
\preprintfooter{FHPC invited talk}   % 'preprint' option specified.

\title{Scalan MetaDSL Framework} % TODO decide final title

\authorinfo{Alexander Slesarenko\and Alexey Romanov}
           {Shannon Laboratory, Huawei Technologies, Moscow, Russia}
           {\{alexander.slesarenko, alexey.romanov\}@huawei.com}

\maketitle

\category{D.3.3}{Programming languages}{Language products and features}

% general terms are not compulsory anymore, 
% you may leave them out
% \terms
% term1, term2

\keywords
DSL, domain-specific languages, high-performance computing, Scalan

\begin{abstract}
While high-level abstractions greatly simplify program development, they ultimately need to be eliminated to produce high-performance code. This can be done using generative programming; one particularly usable approach is Lightweight Modular Staging~\cite{DBLP:journals/cacm/RompfO12, DBLP:conf/snapl/RompfBLSJAOSKDK15}.

We present Scalan, a framework which enables compilation of high-level object-oriented-functional code into high-performance low-level code. It extends the basic LMS approach by making rewrite rules and compilation stages first-class and extending the graph IR with object-oriented features.
% \end{abstract}

Rewrite rules are represented as graph IR nodes with edges pointing to a \emph{pattern graph} and a \emph{replacement graph}; whenever new nodes are constructed, they are compared with the pattern graphs of all active rules and in case a match is found, the corresponding replacement graph is generated instead.

Compilation stages are represented as graph transformers and together with the final output generation stage assembled into a compilation pipeline. This allows using multiple backends together, for example generating C/C++ code with JNI wrappers for the most performance-critical parts and Spark code which calls into it for the rest. % maybe just mention Scala instead of Spark?

We show how object-oriented programming is supported by staging class constructors and method calls (including ``factory'' methods on companion objects) as part of the IR, thus exposing them to rewrite rules like all other operations. JVM mechanisms allow treating symbols as typed proxies for their corresponding nodes. Now it becomes necessary to eliminate such nodes at some compilation stage to avoid virtual dispatch in the output code (or at least minimize it for OO target languages).

In the simple case when the receiver node of a method is a class constructor, we can simply delegate the call to the subject at that stage. The more interesting case when the receiver node is the result of a calculation is handled by \emph{isomorphic specialization}, as previously described in~\cite{DBLP:conf/icfp/SlesarenkoFR14}. % We will explain changes in our implementation since that work.

We will also demonstrate how the use of a Scala compiler plugin further simplifies development by avoiding the explicit use of the {\tt Rep} type constructor and how our framework can handle effects using free monads.
% useful? 

We will finish by discussing future plans for Scalan development.
% Higher kinds? The example is impressive and it wasn't in the previous paper

\end{abstract}

%
%\appendix
%\section{Appendix Title}
%
%This is the text of the appendix, if you need one.
%
%\acks
%
%Acknowledgments, if needed.

% We recommend abbrvnat bibliography style.

\bibliographystyle{abbrvnat}

% The bibliography should be embedded for final submission.

%\bibliography{scalan}

\begin{thebibliography}{3}
\softraggedright
\providecommand{\natexlab}[1]{#1}
\providecommand{\url}[1]{\texttt{#1}}
\expandafter\ifx\csname urlstyle\endcsname\relax
  \providecommand{\doi}[1]{doi: #1}\else
  \providecommand{\doi}{doi: \begingroup \urlstyle{rm}\Url}\fi

\bibitem[Rompf and Odersky(2012)]{DBLP:journals/cacm/RompfO12}
T.~Rompf and M.~Odersky.
\newblock Lightweight modular staging: a pragmatic approach to runtime code
  generation and compiled {DSLs}.
\newblock \emph{Commun. {ACM}}, 55\penalty0 (6):\penalty0 121--130, 2012.

\bibitem[Rompf et~al.(2015)Rompf, Brown, Lee, Sujeeth, Jonnalagedda, Amin,
  Ofenbeck, Stojanov, Klonatos, Dashti, Koch, P{\"{u}}schel, and
  Olukotun]{DBLP:conf/snapl/RompfBLSJAOSKDK15}
T.~Rompf, K.~J. Brown, H.~Lee, A.~K. Sujeeth, M.~Jonnalagedda, N.~Amin,
  G.~Ofenbeck, A.~Stojanov, Y.~Klonatos, M.~Dashti, C.~Koch, M.~P{\"{u}}schel,
  and K.~Olukotun.
\newblock Go meta! {A} case for generative programming and {DSLs} in
  performance critical systems.
\newblock In \emph{1st Summit on Advances in Programming Languages, {SNAPL}
  2015, May 3-6, 2015, Asilomar, California, {USA}}, pages 238--261, 2015.

\bibitem[Slesarenko et~al.(2014)Slesarenko, Filippov, and
  Romanov]{DBLP:conf/icfp/SlesarenkoFR14}
A.~Slesarenko, A.~Filippov, and A.~Romanov.
\newblock First-class isomorphic specialization by staged evaluation.
\newblock In \emph{Proceedings of the 10th {ACM} {SIGPLAN} workshop on Generic
  programming, {WGP} 2014, Gothenburg, Sweden, August 31, 2014}, pages 35--46,
  2014.

\end{thebibliography}


\end{document}
